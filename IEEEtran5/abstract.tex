\begin{abstract}
Online user behavior has been largely studied in various fields due to the prosperity of Internet. In this paper, we investigate Amazon wishlists, where users keep their desired products for easier access or providing guidelines to gift givers. We collected complete Wishlists of over 30,000 users, by analyzing which we are able to reveal interesting observations regarding to user online shopping behaviors in multiple dimensions. Specifically, we show user behavior from different demographical groups, including gender and geo-locations. We found that males and females have different shopping pattern in terms of price and product types. Besides, we also take timing factors into consideration. Surprisingly we found that unlike traditional walk-in-shop type of shopping, user may not always like to shop online during holidays or weekends. Then we investigate user information exposure in Amazon wishlists. By analyzing human languages in list-descriptions, we illustrate what and to what extent are user personal information exposed to the public. Finally, we demonstrate that information in Wishlists has potential to leak personal information that users did not choose to put on Amazon. We use Support-Vector-Machine (SVM) to train and test the data. Our result indicates that based solely on the items in users' Wishlists, We can predict user gender with over 80\% accuracy.

\end{abstract}