\section{Introduction}
Entering the era of big-data, Internet has been largely extended to be data-rich. Network users are prone to expose their private information in public websites inadvertently \cite{frankowski2006you}, making themselves face the threat of information leakage. To study how the information bundle can be used to threat user privacy, we investigate user profile and their wishlists in Amazon, unveiling its user behaviour exposure threat and potential privacy leakage. 

We collected complete profile and wishlists information of over 30,000 users in Amazon by web scraping. Based on the data, we conduct measurement study on user behavior. Our measurement study concentrates on user shopping preference, which are organized in 3 dimensions — 1)What to buy 2)When to buy 3)Product prices users are willing to pay. Specifically, we compare the user preference in different regions and gender. We also investigate shopping peaks and pits through a year. Surprisingly we found that there is little shopping increase in holidays and not all holidays are appealing to shoppers. We believe our findings provide insight that is able to help sellers to revise their marketing strategies, as well as to help advertisers to make more accurate targeted advertising. 

Beside user behavior analysis, we conduct a preliminary study on user privacy information identification using wishlists. We prove that wishlists have potential to leak user personal information. We used the well-know machine learning algorithm -- Support vector machine (SVM) to train and test data. Ground truth are retrieved from user wishlist descriptions in their Amazon profiles. based on the types of products in user wish lists, we are able to identify the user gender with over 70\% success rate, which is substantially more than guessing. 
