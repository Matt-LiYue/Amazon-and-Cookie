\section{Related Work}

There have been papers working on analyzing data and present interesting observations based on the data. For example, Traud et al.~\cite{traud2012social} studied social graph of 100 colleges and universities on Facebook and showed interesting observations such as different institutions have different characteristics. \cite{Lee:2010:USS:1835449.1835522} does statistical analysis of the properties of spam profiles collected from social network communities for creating spam classifiers to actively filter out existing and new spammers. \cite{dave2003mining} describes the system infrastructure, identify the unique properties of list of product attributes and develops a method for automatically distinguishing between positive and negative reviews in Movie Lens to link forum posts to mentioned items. 

When coming to e-commercial data measurement, Mikians et al.~\cite{mikians2012detecting} explored price discrimination problem on several electronic commericals such as Amazon and staples, finding that the same item may have different prices in different regions. Especially some works focus on analyzing user input text such as reviews. Ghose et al.~\cite{ghose2011estimating} studied the review text of several hundred most popular products and explored its impact on economic outcomes such as sales on Amazon. Similarly, Ivanova et al.~\cite{ivanova2013does} studied the review system in Amazon, revealing that user purchasing intention is greatly impacted by product reviews. 

Online privacy is a major user concern. However, users may still leak or expose their data on websites inadvertently. Friedlan et al.~\cite{friedland2010cybercasing} illustrated that users are often unaware the privacy implication of publishing locations. Even worse, users do not even know they published such sensitive information. However, when users realized the privacy implication, their shopping behaviors change significantly. Brown et al.~\cite{brown2004investigating} shows privacy invasion puts significant negative impact on online purchasing behaviors and Tsai et al.~\cite{tsai2011effect}shows that users are more willing to purchase in privacy protective websites if privacy information is salient.

Besides, it has been proven that inferring user personal information based on other data is practical. Using easily accessible public data, user privacy is under huge threat. Narayanan and Shmatikov~\cite{narayanan2009anonymizing} created a new framework to De-anonymize users based on social network topology. Wondracek et al. \cite{wondracek2010practical} leveraged user group membership to uniquely identify an individual user or at least largely reduce candidates. Chaabane et al.~\cite{chaabane2012you} studied the privacy leakage through user interest in music. They can infer user personal information such as gender, age, location, etc based on user self-declared interest. Hecht et al.~\cite{hecht2011tweets} derives user geo-locations from their tweets through machine learning. Goga et al.~\cite{goga2013exploiting} correlates features such as timestamp and writing style of user posts on different websites to identify same user.



