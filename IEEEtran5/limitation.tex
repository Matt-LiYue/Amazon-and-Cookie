\section{Limitation and Future Work}
\subsection{Limitation}
There are limitations in our work. We discuss the limitations and possible consequences from these limitations. 
\begin{enumerate}
\item The data may be biased. The user wish lists are not always public -- users have the ability to change the accessibility of their wish lists (although the default is public). Therefore privacy-aware users may choose to publicize some of their wish lists while keep other wish lists from strangers. Besides, users may choose not to share certain items in their wish lists. For example, privacy-sensitive items such as pregnancy test, firearm-related products, and medicine \& drugs. In our work we can only retrieve the product in public wish lists. The data may not be accurately representative of one users' shopping behavior.

\item When data of price is involved, some products are ignored. We have mentioned that the price cannot always be retrieved. There are multiple reasons and some of them cannot be easily solved such as price not being displayed or javascript being used for displaying the price. Besides, some prices do not accurately reflect the user's preference because there might be multiple prices (from various retailers, used ones or new ones), and we do not know which one does the user prefer. We take only the new one and the first retailer's price. 

\item When doing personal information identification. We actually do not have complete ground truth. Using the keyword approach will give us a group of people with certain features. However, we cannot identify a large enough group without the features. For example, searching "engineer" gives us a group of engineers, however, we do not know whether the users not mentioning "enginner" in their list-descriptions are engineers or not. This lack of complete ground truth may have big negative impact on the success rate of our information identification step.

\end{enumerate}
\subsection{Future Work}
We discuss future work that can be done.
\begin{enumerate}
\item Use possibly better information retrieval method. In our project, the information retrieval step is straight-forward -- we search the keywords in users' list-descriptions to label the users. It works with relatively high accuracy. However, this method is a little too strict so that not many users are selected for a certain class. One way to improve our result is to use a better information retrieval method. We can possibly obtain more users for a specific group while maintaining relatively high accuracy. It will provide more stable data and probably more well-trained SVM.
\item There are 2 ways to improve the user personal information identification process. The first way is to improve the accuracy of the classification step. The second way is to use our method as a filter to filter out irrelevant users and use some other strategies to do further identification.
\end{enumerate}
