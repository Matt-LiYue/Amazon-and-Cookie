\section{Conclusion}
In this paper, we investigate Amazon wishlists, where users store their desired products. We collect over 30,000 users' complete wishlsits and took a 2-step approach to analyze our data. First we try to measure the user behavior from their wish lists by analyzing wishlists in multiple dimensions. Our result shows that there is huge discrepancy between male and female online shopping pattern. The 2 groups differ in both preferred product categories and prices. Furthermore, we investigate user shopping preference taking time as a factor. We reveal that users are in fact not shopping significantly more during holidays or weekends in terms of online shopping. We also investigate different holidays that are shopping appealing or repelling. We highlight that although there are certain holidays are very shopping involving, over half of the national holidays have no increment or even lower shopping favor than normal days. After studying user shopping pattern, we try to identify personal information from list-descriptions or items in the lists. We apply simple human language processing on the plaintext in list-descriptions to generalize the information users tend to expose. Beside the information users mentioned themselves, we also explore the possibility to infer user personal information based on their wishlists. To this end, we selected representative features from wishlists and use SVM to infer user gender. Our results show that given a user that has abundant wishlists, we can predict the user's gender with fairly high accuracy.

